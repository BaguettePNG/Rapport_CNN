\section{Conclusion}

Ce projet de \textbf{Réception LIN} en conception de circuit numérique nous a permis de mettre en œuvre une démarche complète, depuis l’analyse du cahier des charges jusqu’à l’implémentation finale sur FPGA.  

Au cours des différentes étapes, nous avons appliqué une méthodologie rigoureuse basée sur le \textbf{diagramme en Y}, ce qui nous a permis de structurer efficacement notre réflexion et de maîtriser chaque niveau de conception.  
\newline

La phase de \textbf{spécification fonctionnelle} a défini les ressources nécessaires et les signaux principaux.  
La \textbf{solution architecturale} a ensuite permis de modéliser clairement le système à l’aide de machines séquentielles (Moore et Mealy), en distinguant la partie commande et la partie opérative.  
Cette approche a facilité l’écriture d’un code \textbf{VHDL clair, modulaire et cohérent}.  
\newline

Les phases de \textbf{simulation} et de \textbf{synthèse} ont confirmé la validité du fonctionnement logique et le bon découpage du système en ressources matérielles (LUT, bascules, multiplexeurs, etc.).  
Enfin, l’\textbf{implémentation sur FPGA} a validé le fonctionnement global du récepteur LIN, en assurant le respect des contraintes temporelles et la conformité au cahier des charges.  
\newline

Ainsi, ce projet nous a permis de concevoir et de valider avec succès un système complet de \textbf{réception LIN}, pleinement opérationnel et prêt à être intégré dans un environnement embarqué.  
Il nous a également apporté une meilleure maîtrise des outils de conception numérique (\textit{Vivado}, simulations \textit{VHDL}) et une compréhension approfondie du lien entre la modélisation logique et la réalisation matérielle.  
